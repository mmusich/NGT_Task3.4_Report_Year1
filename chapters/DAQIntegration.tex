\section{DAQ Integration} % Thiago

%%% We just write the discussion we had in the whole DAQ session

The Optimal Calibration system has to be embedded in the CMS trigger and data acquisition (DAQ) system,
both in the demonstrator system and in the Phase-2 implementation.
We briefly discuss the DAQ system architecture implemented for Run 3, 
and how the demonstrator can be inserted in that system.
We then review the conceptual DAQ design proposed for Phase-2, and 
go through a preliminary calculation of the NGT system feasibility.

\subsection{Demonstrator Integration in Run 3}

The CMS DAQ system for Run 3 (DAQ3) is an intricate system, dealing with all steps of the data flow from the sub-detector front-end drivers (FEDs) through the final delivery of data streams to Tier-0. For the purposes of our discussions, we can focus on the following elements of Figure~\ref{fig:DAQ3}: 
the \emph{Readout Units} (RUs) unpack and merge event fragments into super-fragments, which are then sent to the \emph{Builder Units} (BUs) to be further assembled into complete events. 
The complete events are sent to the \emph{Filter Units} (FUs), where the HLT application runs and selects a subset of the events to be saved for permanent storage. 
In the DAQ3 implementation, the RUs and the BUs are logically separate, but physically realised in the same computer node.

The output data are then sent back to the BUs, where they are further merged and their management is handed over to the storage and transfer system (STS).
The STS  transfer the data to a (Lustre-based) cluster file system, whence they are finally sent to their final destination:
Tier-0,
DQM
or the calibration cluster.
The DAQ3 was also designed with two characteristics that the demonstrator should respect: 
i) it is \emph{pipelined}, so the data should follow an unidirectional flow;
ii) it is \emph{democratic}, in the sense that all elements (RUs, BUs, FUs) are treated equally.

%%% Thiago testing the GitHub integration
% I want to work at the LHC!
% I want to work at the FCC!